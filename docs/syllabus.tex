
\documentclass[12pt]{amsart}
\usepackage{geometry} % see geometry.pdf on how to lay out the page. There's lots.
\geometry{a4paper} % or letter or a5paper or ... etc
\usepackage[T1]{fontenc}
\usepackage[latin9]{inputenc}
%\usepackage{listings}
\usepackage{enumerate}
\usepackage{enumitem}
\usepackage{multirow}
\usepackage{colortbl}
\usepackage{amssymb}
\usepackage[colorlinks = true,
            linkcolor = blue,
            urlcolor  = blue,
            citecolor = blue,
            anchorcolor = blue]{hyperref}
\usepackage{makecell}
\usepackage{pifont}
\usepackage{float}
\usepackage{listings}
\usepackage{color}
 
\definecolor{codegreen}{rgb}{0,0.6,0}
\definecolor{codegray}{rgb}{0.5,0.5,0.5}
\definecolor{stringcolor}{rgb}{0.7,0.23,0.36}
\definecolor{backcolour}{rgb}{0.95,0.95,0.92}
\definecolor{keycolor}{rgb}{0.007,0.01,1.0}
\definecolor{itemcolor}{rgb}{0.01,0.0,0.49}
 
\lstset{
 basicstyle=\ttfamily,
 columns=fullflexible,
 upquote,
 keepspaces,
 literate={*}{{\char42}}1
 {-}{{\char45}}1
}

\title{CSE 481 Syllabus}

\date{\today}



\makeatletter
\newcommand{\skipitems}[1]{%
  \addtocounter{\@enumctr}{#1}%
}
\makeatother

%%% BEGIN DOCUMENT
\begin{document}
\maketitle
This course introduces you to how to design, write, and scale massively concurrent distributed systems. Systems that can handle millions of requests per second.


\section{Objectives}
\begin{itemize}
	\item Make rational, sensible choices between differing ways to compute solutions to problems.
	\item Apply the principles of feedback and scalability when designing computations.
	\item Create rationally designed, massively parallel cloud systems.
	\item Create a secure cloud space using linux, firewalls, https certificates, and ssh.
	\item Create virtual linux instances modified in such a way that they support very high throughput.
	\item Have an increased understanding of how parallelism and functional programming is used in industry.
	\item Apply generative AI in an industry appropriate way.

\end{itemize}

\section{Prerequisites}

  You must have successfully completed the following courses:
\begin{itemize}
    \item CSE 121e Erlang Language
    \item CSE 382 Patterns and Data Structures
\end{itemize}
  It is very strongly suggested that you have successfully completed the following courses:
\begin{itemize}
    \item CSE 381 Algorithms and Complexity
    \item CSE 251 Parallelism and Concurrency
\end{itemize}
  You must have \textit{some} working knowledge of:
\begin{itemize}
     \item Computational parallelism and concurrency
     \item Basic data structures (sets, lists, maps, trees, graphs, etc.)
\end{itemize}

\section{Requirements}
You are required to obtain
\subsection{Optional Texts}
	\begin{itemize}
	\item \textbf{Programming Erlang: Software for a Concurrent World}.\textit{Joe Armstrong},$2^{nd}$ Edition, Pragmatic Programmers
	\end{itemize}
	This is available online and for free through the BYUI library. Do a Search for it. You can also choose to purchase it online if you want a hard copy.
\subsection{Documents} As provided by the instructor.
\subsection{Software}
	\begin{itemize}
		\item The Erlang compiler and runtime
		\item Rebar3
		\item Cowboy
		\item Riak
		\item A text editor of your choice
	\end{itemize}
\subsection{Hardware}
	\begin{itemize}
		\item A laptop that can run the required software reasonably fast.
		\item One or more Digital Ocean Droplets and other DO components. (Instructor will show you how to get them for free.)
	\end{itemize}
\section{Behavioral Requirements}

You are required to\ldots
\begin{itemize}
\item attend class, as personal assessments will happen in class each day that are not reproducible outside of class.
\item Use the provided Learning Assistants to help you learn the concepts of OTP and concurrency.
\item complete all tasks as part of a team in order to deepen your understanding of selected topics.
\end{itemize}

\subsection{Late Work} There are no assignments, quizzes, or exams that can be late.

\subsection{Grades} In each of our three personal meetings, you will present your course and project trackers. You will also present a letter based grade-to-date claim. Afterwards I will give you my thoughts on the strength of your claim. The last claim that you make, taking into account any feedback from me, will be your final grade for the course. All of your claims must must be evidence based. That means you must bring the evidence with you, in your portfolio, that supports your claim. 
\subsection{Letter-Based-Grades}You are required to use the definition of the grades from the University Catalog:
\begin{enumerate}[label=\textbf{\Alph*}]
	\item represents outstanding understanding, application, and integration of subject material and extensive evidence of original thinking, skillful use of concepts, and ability to analyze and solve complex problems. Demonstrates diligent application of Learning Model principles, including initiative in serving other students.
	Note: To claim this grade, throughout the 4 week period being reviewed, you \textit{must} have consistently done things similar to what you see in the list below  and recorded evidence of this behavior in your portfolio. 
Examples of the required types of behaviors are:
\begin{itemize}  
\item teaching and/or helping others in the class but not in your group,
\item applying what you've learned in this class in your job or another class you are currently taking, and
\item doing work not assigned such as writing code using what you are learning that has not been assigned, etc.
\item seeing what you are learning around you in the real, not computing, world weekly-ish.
\end{itemize}

	\item represents considerable/significant understanding, application, and incorporation of the material which would prepare a student to be successful in next level courses, graduate school or employment. The student participates in the Learning Model as applied in the course. 
	\item represents sufficient understanding of subject matter. The student demonstrates minimal initiative to be prepared for class. Sequenced courses could be attempted, but mastering new materials might prove challenging. The student participates only marginally in the Learning Model.
	\item represents poor performance and initiative to learn and understand and apply course materials. Retaking a course or remediation may be necessary to prepare for additional instruction in this subject matter. 
\end{enumerate}

\section{University Policies}
To review University policies regarding disabilities, sexual harassment, etc., or to arrange for a tutor from the Academic Support Center, select \textbf{Modules} in the iLearn course, scroll to the Student Resources module, and select the appropriate link.

\section{Other}
This document may be modified by the instructor at any time without notification.
\end{document}